%%%% ijcai26.tex

\typeout{IJCAI--ECAI 26 Instructions for Authors}

% These are the instructions for authors for IJCAI--ECAI 26.

\documentclass{article}
\pdfpagewidth=8.5in
\pdfpageheight=11in
\usepackage[UTF8]{ctex} % 允许在文档中使用中文
% The file ijcai26.sty is a copy from ijcai22.sty
% The file ijcai22.sty is NOT the same as previous years'
\usepackage{ijcai26}

% Use the postscript times font!
\usepackage{times}
\usepackage{soul}
\usepackage{url}
\usepackage[hidelinks]{hyperref}
\usepackage[utf8]{inputenc}
\usepackage[small]{caption}
\usepackage{graphicx}
\usepackage{amsmath}
\usepackage{amsthm}
\usepackage{booktabs}
\usepackage{algorithm}
\usepackage{algorithmic}
\usepackage[switch]{lineno}

% Comment out this line in the camera-ready submission
\linenumbers

\urlstyle{same}

% the following package is optional:
%\usepackage{latexsym}

% See https://www.overleaf.com/learn/latex/theorems_and_proofs
% for a nice explanation of how to define new theorems, but keep
% in mind that the amsthm package is already included in this
% template and that you must *not* alter the styling.
\newtheorem{example}{Example}
\newtheorem{theorem}{Theorem}

% Following comment is from ijcai97-submit.tex:
% The preparation of these files was supported by Schlumberger Palo Alto
% Research, AT\&T Bell Laboratories, and Morgan Kaufmann Publishers.
% Shirley Jowell, of Morgan Kaufmann Publishers, and Peter F.
% Patel-Schneider, of AT\&T Bell Laboratories collaborated on their
% preparation.

% These instructions can be modified and used in other conferences as long
% as credit to the authors and supporting agencies is retained, this notice
% is not changed, and further modification or reuse is not restricted.
% Neither Shirley Jowell nor Peter F. Patel-Schneider can be listed as
% contacts for providing assistance without their prior permission.

% To use for other conferences, change references to files and the
% conference appropriate and use other authors, contacts, publishers, and
% organizations.
% Also change the deadline and address for returning papers and the length and
% page charge instructions.
% Put where the files are available in the appropriate places.


% PDF Info Is REQUIRED.

% Please leave this \pdfinfo block untouched both for the submission and
% Camera Ready Copy. Do not include Title and Author information in the pdfinfo section
\pdfinfo{
/TemplateVersion (IJCAI.2026.0)
}

\title{Rule-Guided Reasoning via Multi-Agent Test-Driven Program Synthesis: A Neuro-Symbolic Approach
}

% Single author syntax
\author{
    Author Name
    \affiliations
    Affiliation
    \emails
    email@example.com
}

% Multiple author syntax (remove the single-author syntax above and the \iffalse ... \fi here)
\iffalse
\author{
First Author$^1$
\and
Second Author$^2$\and
Third Author$^{2,3}$\And
Fourth Author$^4$\\
\affiliations
$^1$First Affiliation\\
$^2$Second Affiliation\\
$^3$Third Affiliation\\
$^4$Fourth Affiliation\\
\emails
\{first, second\}@example.com,
third@other.example.com,
fourth@example.com
}
\fi

\begin{document}

\maketitle

\begin{abstract}
Large Language Models (LLMs) often struggle with complex rule-guided reasoning tasks due to challenges in rule recall, logic consistency, and precise numerical computation. To address these limitations, we propose a novel multi-agent framework that transforms natural language rules into executable and verifiable program segments. Our approach employs a rule-decomposition mechanism to break down complex instructions into atomic, manageable sub-rules. We then introduce a collaborative multi-agent system comprising Code Generation, Test Case Generation, and Verification agents. By adopting a test-driven development (TDD) paradigm, the framework iteratively refines the synthesized code based on execution feedback from a symbolic executor, ensuring strict adherence to the underlying rules. Experimental results demonstrate that our method significantly reduces rule omission and logical errors compared to baselines such as directly prompting LLMs for results and code generation under full rule sets, providing a transparent and traceable reasoning process for complex real-world scenarios.
\end{abstract}

\section{Introduction}
\subsection{研究背景}

近年来,大语言模型(LLMs)在自然语言处理领域展现出了卓越的能力,但在面对现实世界中复杂的规则引导推理(Rule-Guided Reasoning)时,依然表现出明显的局限性。以 RULEARENA 数据集中的场景为例,无论是计算航空公司行李费、判断NBA球员交易是否符合劳资协议,还是处理严苛的税务审计计算,都需要模型具备极高的逻辑严密性和计算精确度。在这些现实场景中,任何微小的规则理解偏差或计算错误都可能导致错误,因此,研究如何提升模型在复杂规则下的推理能力具有极重要的意义。

\subsection{纯神经推理的局限性}

为了推动这一领域的研究,本文深入探讨了包含上述三个领域的复杂规则数据集。然而,实验发现现有的最先进模型(SOTA)在这些任务上的表现均不尽如人意。通过分析,我们总结出其失败的核心原因在于:模型没有对必须复杂推理任务实现规则的精准理解与绝对遵守。具体表现为以下几点:

规则规模庞大:如税务相关规则有上千行,模型难以在长上下文中保持关注;

规则记忆与幻觉:根据我们的实验分析,随着规则数量增加,模型会对规则出现明显的记忆偏差,从而降低解答规则所对应的问题的准确率。

理解偏差:纯神经模型在解析自然语言规则时,经常产生逻辑偏差,往往需要繁琐的人工干预才能纠正。

\subsection{神经符号方法}

现有的纯生成式神经方法(Neural-only)本质上是基于概率的预测,很难在处理确定性极强的逻辑规则时保证零误差。因此,我们认为将符号系统(Symbolic)引入推理过程是必然选择。

规则天然地可以转化为确定性的编程代码或逻辑符号。在符号表达正确的前提下,程序执行可以完全避免神经模型在推理过程中的记忆损耗和随机性幻觉。基于此,我们提出了一种神经+符号(Neuro-Symbolic)多智能体框架。在该系统中,神经智能体负责将复杂的自然语言规则解析并转化为可执行的符号代码,而符号智能体负责执行逻辑并反馈结果。

这种神经符号协作模式的优势在于:它将 LLM 从沉重的逻辑计算中解放出来,让其回归到擅长的语义理解上,而将严密的逻辑推导交给稳定、透明且可解释的代码系统。

\subsection{分步生成与 Case 驱动校验}

然而,实现这种结合面临巨大的挑战:如何确保 LLM 生成的代码(符号)是完全正确的? 为此,我们提出了基于逻辑分割的步骤化生成与自动边界 Case 校验机制。系统会要求模型自动根据规则生成一系列临界情况(Boundary Cases)作为验证集,并在代码空间中进行迭代搜索和调试。只有通过了全量校验的代码才会被视为最终的推理工具,从而在根本上保证了符号的正确性。

实验结果表明,该机制显著提升了符号代码的生成质量。在与 GPT-4o 和 Qwen-2.5 72B 的对比实验中,我们的方法取得了性能提升。例如,在挑战性极大的 Airline 任务中,GPT-4o 在全量规则下的准确率几乎为 0,而本方法将其提升至了 0.83。

本文的主要贡献总结如下:

提出了一个端到端的神经+符号多智能体推理框架,有效解决了 LLM 在复杂规则下的幻觉问题;

引入了分步生成与自动 Case 校验机制,利用模型自生成的验证集在代码空间进行搜索,确保了符号逻辑的确定性;

在多个现实世界复杂任务上取得了显著的性能提升,验证了框架的鲁棒性和实用性。


\section{Related Work}

\subsection{复杂指令与规则引导推理}

大模型在处理包含多个约束的复杂指令时的表现已成为当前研究的热点。ComplexBench \cite{wen:2024:complexbench} 提出了一种层次化的分类法,通过组合多种约束维度来评估模型的指令遵循能力。在更深层次的逻辑约束方面,RuleEval \cite{sun:2024:ruleeval}进一步区分了通用指令遵循与推理规则遵循,研究了模型在规则冲突和反事实设置下的鲁棒性。此外,为了增强模型处理高复杂度任务的能力,WizardLM \cite{xu:2023:wizardlm}通过 Evol-Instruct 方法自动生成复杂指令来微调模型,而 Counterfactual Reasoning \cite{wang:2023:counterfactual}则通过改变基础规则(如改变进制或算术符号)来测试模型是真正具备推理能力还是仅仅在背诵训练数据。

\subsection{神经符号结合的推理}

由于纯神经模型在长链条逻辑推理中容易产生幻觉,将符号求解器与 LLM 结合成为一种趋势。LOGIC-LM \cite{pan:2023:logiclm}提出了一个标准框架,利用 LLM 将自然语言翻译为符号公式(如一阶逻辑),并交由确定性求解器(如 Z3 或 Pyke)处理,从而保证了逻辑的严密性。ProofWriter \cite{afjord:2021:proofwriter}则探索了模型生成证明链的能力,通过迭代生成一阶段蕴含(implication)来构建完整的逻辑证明。这类工作证明了符号逻辑系统在修正神经模型逻辑缺陷方面的巨大潜力。

\subsection{智能体协作与工具增强}

多智能体框架通过角色分工进一步提升了复杂问题的处理效率。DyLAN \cite{liu:2023:dylan}提出了动态智能体网络,根据任务贡献度自动筛选智能体团队进行协作。在工具利用方面,ChatCoT \cite{chen:2023:chatcot}将思维链(CoT)推理建模为多轮对话,使模型能以更自然的方式调用外部工具(如 Python 解释器)。这些框架为复杂规则下的分工协作提供了范式参考。

\subsection{本工作与现有工作的区别}

\subsubsection{场景与数据集}
现有工作如 ComplexBench 或 RuleEval 多侧重于人工构建的、规则条数较少的通用逻辑约束。相比之下,我们的 RULEARENA 数据集来源于真实的复杂领域(如 NBA 劳资协议、美国税务法规和民航行李费),其规则体量大(通常包含数百条相互关联的条款)、专业性强,对模型的长上下文管理和精准规则定位能力提出了更高挑战。

\subsubsection{方法与框架}
不同于 LOGIC-LM 等依赖单一转换的工作,我们提出了一个神经+符号的多智能体系统。该框架的核心创新在于引入了自动边界 Case 校验机制:针对复杂规则中难以通过人工覆盖的逻辑死角,系统能够自动生成边界测试案例作为“验证集”,对生成的符号代码进行在线回测和修正。这种闭环校验机制弥补了现有神经符号系统中“模型翻译代码后缺乏自审”的缺陷,显著提升了在严苛规则场景下的推理准确率。


\section{Methodology}
本研究的核心在于将不稳定的概率性神经推理转换为确定性的符号执行推理。我们不要求模型直接给出答案,而是要求其构建一套能够计算出答案的软件系统。

\subsection{任务建模与形式化定义}
我们将复杂规则下的推理任务定义为三元组 (R,P,A)。

R (Rules):由数千字构成的自然语言规则集,包含显式定义的法律条文或行业规定,以及隐式存在的逻辑结构。

P (Problem):包含具体数值、实体和约束条件的待解问题。

A (Answer):在 R 约束下,针对 P 产生的唯一合法结果。

传统的推理范式 LLM(R,P)→A 在 R 的长度增加时,会因为注意力机制的稀释和上下文窗口的逻辑拥挤而产生幻觉。本方法引入符号层 S(Python 脚本),将过程转化为:

分段规则解构:将 R 拆解为 $\{r_1, r_2, \dots, r_n\}$

符号逻辑构建:对于每个子任务,通过 MAS 生成相应的符号函数 $f_i$。

确定性执行:最终结果 $A$ 通过串联执行函数流 $\prod_{i=1}^{n} f_i(P)$ 获得。

\subsection{基于 AutoGen 的多智能体协作}
我们利用 AutoGen 框架构建了一个闭环的“开发-测试-修正”环境。

\subsubsection{基于 AutoGen 的多智能体协作}
负责核心算法的编写。它根据 $r_i$ 和输入格式示例(Input Example),利用正则表达式解析输入数据,并构建复杂的逻辑判断分支。其工具链用于在本地文件系统中持久化符号逻辑。

\subsubsection{测试样例设计智能体 (Sample Agent)}
它根据代码逻辑推导潜在的失败路径,生成包含正常值、边界值(如刚好达到税率阶梯点或行李尺寸临界值)的测试字典,作为 Code Agent 的压力测试集。

\subsubsection{代码测试智能体 (Tester Agent)}
系统驱动核心。它实时运行 Sample Agent 提供的测试用例。

\subsection{核心机制:物理反馈与自修复循环}
与传统生成方式不同,本系统具备真实的物理反馈。当代码运行失败时,Tester Agent 捕获的错误日志(如 KeyError、TypeError 或 Pydantic 验证错误)会作为指令返回给 Code Agent。这种“在代码空间中搜索最优解”的过程,能够自动修正由于 LLM 理解偏差导致的逻辑错误。当 Tester Agent 发现逻辑偏差(如航线超限费用计算为 0,实际应为 200)时,系统会启动新一轮的对话。Code Agent 会重新审视单位换算(如英寸与厘米的混淆)并重构逻辑。这种在代码搜索空间(Code Space Search)中的迭代,极大提高了系统处理复杂规则的一致性。

\section{Experiments}

\subsection{实验设置与数据集}

我们在三个具有代表性的复杂规则领域开展了主实验:

Airline (航空行李费): 涉及复杂的航线区域映射与超重/超尺寸阶梯计费,任务拆分为 9 步。

Tax (美国税务): 包含多表关联计算与累进税率逻辑,任务拆分为 14 步。

NBA (劳资协议交易): 涉及硬帽限制、TPE 匹配等极端复杂的金融与合同约束,任务拆分为 8 步。

\subsection{规则规模对模型能力的影响}

在开展主实验前,我们探究了规则长度对于模型回答效果的简单实验。

设计了Airline数据集上的简单问题,在不断增加干扰规则的情况下考察模型对于相同问题的回答,在不断增加规则长度的情况下要求模型回答规则对应的问题,发现准确率均有一定程度的下降。

对于Airline数据集的前一百个问题,我们通过人工的方式找到了全量规则文本中与回答问题最相关的规则子集。发现在仅提供最相关的规则子集的情况下模型的回答会与真实答案更接近。

这证明了将规则拆解的必要性。

\subsection{主实验结果分析}

我们将本方法与 0-shot 纯推理以及“全部规则不拆分”的基线进行了对比(模型使用 Qwen-2.5 72B 与 GPT-4o)。实验数据显示:

在 Tax 数据集上,本方法在 GPT-4o 下达到了 100% 的准确率,远超直接推理的表现。

在最复杂的 NBA 协议中,本方法通过分段生成 Python 脚本,解决了纯神经模型无法处理的 TPE 三级限额计算等硬逻辑瓶颈。

结论: 实验定性地证明,通过将复杂规则“分而治之”并转化为可执行代码,能有效消除模型在长程推理中的计算漂移。

\subsection{案例分析:从失败到成功的闭环}

\subsection{效率损耗分析与工程瓶颈}

我们对系统的推理损耗进行了定量化评估,并得出以下定性结论:

保险溢价成本:虽然 MAS 方法平均产生了约 8.2 倍的调用损耗,但这一成本换取了一定的逻辑确定性,自动修复了大量因路径解析、API 参数或逻辑细节导致的潜在错误。

工程复杂度瓶颈:实验发现,系统主要的效率损耗并非源于复杂的算法计算(如累进税率),而是源于对“物理世界”的摸索,如处理跨文件引用或沙箱环境下的权限限制。

协作演化趋势:随着推理步骤的推进,智能体表现出角色合并的倾向。Code Agent 会自发承担部分测试代码的编写工作,使系统损耗趋于一个稳定的“地板水平”(约 6 次调用),展现了多智能体协作的自适应优化能力。


\begin{itemize}
\item If your track requires submissions to be anonymous, they must be fully anonymized as discussed in the Modifications for Blind Review subsection below; in this case, Acknowledgements and Contribution Statement sections are not allowed.

\item If your track requires non-anonymous submissions, you should provide all author information at the time of submission, just as for camera-ready papers (see below); Acknowledgements and Contribution Statement sections are allowed, but optional.

\item Submissions must include line numbers to facilitate feedback in the review process . Enable line numbers by uncommenting the command {\tt \textbackslash{}linenumbers} in the preamble.

\item The limit on the number of  content pages is \emph{strict}. All papers exceeding the limits will be desk rejected.
\end{itemize}

\subsubsection{Camera-Ready Papers}
The following instructions apply to camera-ready papers:

\begin{itemize}
\item Authors and affiliations are mandatory. Explicit self-references are allowed. It is strictly forbidden to add authors not declared at submission time.

\item Acknowledgements and Contribution Statement sections are allowed, but optional.

\item Line numbering must be disabled. To achieve this, comment or disable {\tt \textbackslash{}linenumbers} in the preamble.

\item For some of the tracks, you can exceed the page limit by purchasing extra pages.
\end{itemize}

\subsection{Title and Author Information}

Center the title on the entire width of the page in a 14-point bold
font. The title must be capitalized using Title Case. For non-anonymous papers, author names and affiliations should appear below the title. Center author name(s) in 12-point bold font. On the following line(s) place the affiliations.

\subsubsection{Author Names}

Each author name must be followed by:
\begin{itemize}
    \item A newline {\tt \textbackslash{}\textbackslash{}} command for the last author.
    \item An {\tt \textbackslash{}And} command for the second to last author.
    \item An {\tt \textbackslash{}and} command for the other authors.
\end{itemize}

\subsubsection{Affiliations}

After all authors, start the affiliations section by using the {\tt \textbackslash{}affiliations} command.
Each affiliation must be terminated by a newline {\tt \textbackslash{}\textbackslash{}} command. Make sure that you include the newline after the last affiliation, too.

\subsubsection{Mapping Authors to Affiliations}

If some scenarios, the affiliation of each author is clear without any further indication (\emph{e.g.}, all authors share the same affiliation, all authors have a single and different affiliation). In these situations you don't need to do anything special.

In more complex scenarios you will have to clearly indicate the affiliation(s) for each author. This is done by using numeric math superscripts {\tt \$\{\^{}$i,j, \ldots$\}\$}. You must use numbers, not symbols, because those are reserved for footnotes in this section (should you need them). Check the authors definition in this example for reference.

\subsubsection{Emails}

This section is optional, and can be omitted entirely if you prefer. If you want to include e-mails, you should either include all authors' e-mails or just the contact author(s)' ones.

Start the e-mails section with the {\tt \textbackslash{}emails} command. After that, write all emails you want to include separated by a comma and a space, following the order used for the authors (\emph{i.e.}, the first e-mail should correspond to the first author, the second e-mail to the second author and so on).

You may ``contract" consecutive e-mails on the same domain as shown in this example (write the users' part within curly brackets, followed by the domain name). Only e-mails of the exact same domain may be contracted. For instance, you cannot contract ``person@example.com" and ``other@test.example.com" because the domains are different.


\subsubsection{Modifications for Blind Review}
When submitting to a track that requires anonymous submissions,
in order to make blind reviewing possible, authors must omit their
names, affiliations and e-mails. In place
of names, affiliations and e-mails, you can optionally provide the submission number and/or
a list of content areas. When referring to one's own work,
use the third person rather than the
first person. For example, say, ``Previously,
Gottlob~\shortcite{gottlob:nonmon} has shown that\ldots'', rather
than, ``In our previous work~\cite{gottlob:nonmon}, we have shown
that\ldots'' Try to avoid including any information in the body of the
paper or references that would identify the authors or their
institutions, such as acknowledgements. Such information can be added post-acceptance to be included in the camera-ready
version.
Please also make sure that your paper metadata does not reveal
the authors' identities.

\subsection{Abstract}

Place the abstract at the beginning of the first column 3$''$ from the
top of the page, unless that does not leave enough room for the title
and author information. Use a slightly smaller width than in the body
of the paper. Head the abstract with ``Abstract'' centered above the
body of the abstract in a 12-point bold font. The body of the abstract
should be in the same font as the body of the paper.

The abstract should be a concise, one-paragraph summary describing the
general thesis and conclusion of your paper. A reader should be able
to learn the purpose of the paper and the reason for its importance
from the abstract. The abstract should be no more than 200 words long.

\subsection{Text}

The main body of the text immediately follows the abstract. Use
10-point type in a clear, readable font with 1-point leading (10 on
11).

Indent when starting a new paragraph, except after major headings.

\subsection{Headings and Sections}

When necessary, headings should be used to separate major sections of
your paper. (These instructions use many headings to demonstrate their
appearance; your paper should have fewer headings.). All headings should be capitalized using Title Case.

\subsubsection{Section Headings}

Print section headings in 12-point bold type in the style shown in
these instructions. Leave a blank space of approximately 10 points
above and 4 points below section headings.  Number sections with
Arabic numerals.

\subsubsection{Subsection Headings}

Print subsection headings in 11-point bold type. Leave a blank space
of approximately 8 points above and 3 points below subsection
headings. Number subsections with the section number and the
subsection number (in Arabic numerals) separated by a
period.

\subsubsection{Subsubsection Headings}

Print subsubsection headings in 10-point bold type. Leave a blank
space of approximately 6 points above subsubsection headings. Do not
number subsubsections.

\paragraph{Titled paragraphs.} You should use titled paragraphs if and
only if the title covers exactly one paragraph. Such paragraphs should be
separated from the preceding content by at least 3pt, and no more than
6pt. The title should be in 10pt bold font and to end with a period.
After that, a 1em horizontal space should follow the title before
the paragraph's text.

In \LaTeX{} titled paragraphs should be typeset using
\begin{quote}
    {\tt \textbackslash{}paragraph\{Title.\} text} .
\end{quote}

\subsection{Special Sections}

\subsubsection{Appendices}
You may move some of the contents of the paper into one or more appendices that appear after the main content, but before references. These appendices count towards the page limit and are distinct from the supplementary material that can be submitted separately through CMT. Such appendices are useful if you would like to include highly technical material (such as a lengthy calculation) that will disrupt the flow of the paper. They can be included both in papers submitted for review and in camera-ready versions; in the latter case, they will be included in the proceedings (whereas the supplementary materials will not be included in the proceedings).
Appendices are optional. Appendices must appear after the main content.
Appendix sections must use letters instead of Arabic numerals. In \LaTeX, you can use the {\tt \textbackslash{}appendix} command to achieve this followed by  {\tt \textbackslash section\{Appendix\}} for your appendix sections.

\subsubsection{Ethical Statement}

Ethical Statement is optional. You may include an Ethical Statement to discuss  the ethical aspects and implications of your research. The section should be titled \emph{Ethical Statement} and be typeset like any regular section but without being numbered. This section may be placed on the References pages.

Use
\begin{quote}
    {\tt \textbackslash{}section*\{Ethical Statement\}}
\end{quote}

\subsubsection{Acknowledgements}

Acknowledgements are optional. In the camera-ready version you may include an unnumbered acknowledgments section, including acknowledgments of help from colleagues, financial support, and permission to publish. This is not allowed in the anonymous submission. If present, acknowledgements must be in a dedicated, unnumbered section appearing after all regular sections but before references.  This section may be placed on the References pages.

Use
\begin{quote}
    {\tt \textbackslash{}section*\{Acknowledgements\}}
\end{quote}
to typeset the acknowledgements section in \LaTeX{}.


\subsubsection{Contribution Statement}

Contribution Statement is optional. In the camera-ready version you may include an unnumbered Contribution Statement section, explicitly describing the contribution of each of the co-authors to the paper. This is not allowed in the anonymous submission. If present, Contribution Statement must be in a dedicated, unnumbered section appearing after all regular sections but before references.  This section may be placed on the References pages.

Use
\begin{quote}
    {\tt \textbackslash{}section*\{Contribution Statement\}}
\end{quote}
to typeset the Contribution Statement section in \LaTeX{}.

\subsubsection{References}

The references section is headed ``References'', printed in the same
style as a section heading but without a number. A sample list of
references is given at the end of these instructions. Use a consistent
format for references. The reference list should not include publicly unavailable work.

\subsubsection{Order of Sections}
Sections should be arranged in the following order:
\begin{enumerate}
    \item Main content sections (numbered)
    \item Appendices (optional, numbered using capital letters)
    \item Ethical statement (optional, unnumbered)
    \item Acknowledgements (optional, unnumbered)
    \item Contribution statement (optional, unnumbered)
    \item References (required, unnumbered)
\end{enumerate}

\subsection{Citations}

Citations within the text should include the author's last name and
the year of publication, for example~\cite{gottlob:nonmon}.  Append
lowercase letters to the year in cases of ambiguity.  Treat multiple
authors as in the following examples:~\cite{abelson-et-al:scheme}
or~\cite{bgf:Lixto} (for more than two authors) and
\cite{brachman-schmolze:kl-one} (for two authors).  If the author
portion of a citation is obvious, omit it, e.g.,
Nebel~\shortcite{nebel:jair-2000}.  Collapse multiple citations as
follows:~\cite{gls:hypertrees,levesque:functional-foundations}.
\nocite{abelson-et-al:scheme}
\nocite{bgf:Lixto}
\nocite{brachman-schmolze:kl-one}
\nocite{gottlob:nonmon}
\nocite{gls:hypertrees}
\nocite{levesque:functional-foundations}
\nocite{levesque:belief}
\nocite{nebel:jair-2000}

\subsection{Footnotes}

Place footnotes at the bottom of the page in a 9-point font.  Refer to
them with superscript numbers.\footnote{This is how your footnotes
    should appear.} Separate them from the text by a short
line.\footnote{Note the line separating these footnotes from the
    text.} Avoid footnotes as much as possible; they interrupt the flow of
the text.

\section{Illustrations}

Place all illustrations (figures, drawings, tables, and photographs)
throughout the paper at the places where they are first discussed,
rather than at the end of the paper.

They should be floated to the top (preferred) or bottom of the page,
unless they are an integral part
of your narrative flow. When placed at the bottom or top of
a page, illustrations may run across both columns, but not when they
appear inline.

Illustrations must be rendered electronically or scanned and placed
directly in your document. They should be cropped outside \LaTeX{},
otherwise portions of the image could reappear during the post-processing of your paper.
When possible, generate your illustrations in a vector format.
When using bitmaps, please use 300dpi resolution at least.
All illustrations should be understandable when printed in black and
white, albeit you can use colors to enhance them. Line weights should
be 1/2-point or thicker. Avoid screens and superimposing type on
patterns, as these effects may not reproduce well.

Number illustrations sequentially. Use references of the following
form: Figure 1, Table 2, etc. Place illustration numbers and captions
under illustrations. Leave a margin of 1/4-inch around the area
covered by the illustration and caption.  Use 9-point type for
captions, labels, and other text in illustrations. Captions should always appear below the illustration.

\section{Tables}

Tables are treated as illustrations containing data. Therefore, they should also appear floated to the top (preferably) or bottom of the page, and with the captions below them.

\begin{table}
    \centering
    \begin{tabular}{lll}
        \hline
        Scenario  & $\delta$ & Runtime \\
        \hline
        Paris     & 0.1s     & 13.65ms \\
        Paris     & 0.2s     & 0.01ms  \\
        New York  & 0.1s     & 92.50ms \\
        Singapore & 0.1s     & 33.33ms \\
        Singapore & 0.2s     & 23.01ms \\
        \hline
    \end{tabular}
    \caption{Latex default table}
    \label{tab:plain}
\end{table}

\begin{table}
    \centering
    \begin{tabular}{lrr}
        \toprule
        Scenario  & $\delta$ (s) & Runtime (ms) \\
        \midrule
        Paris     & 0.1          & 13.65        \\
                  & 0.2          & 0.01         \\
        New York  & 0.1          & 92.50        \\
        Singapore & 0.1          & 33.33        \\
                  & 0.2          & 23.01        \\
        \bottomrule
    \end{tabular}
    \caption{Booktabs table}
    \label{tab:booktabs}
\end{table}

If you are using \LaTeX, you should use the {\tt booktabs} package, because it produces tables that are better than the standard ones. Compare Tables~\ref{tab:plain} and~\ref{tab:booktabs}. The latter is clearly more readable for three reasons:

\begin{enumerate}
    \item The styling is better thanks to using the {\tt booktabs} rulers instead of the default ones.
    \item Numeric columns are right-aligned, making it easier to compare the numbers. Make sure to also right-align the corresponding headers, and to use the same precision for all numbers.
    \item We avoid unnecessary repetition, both between lines (no need to repeat the scenario name in this case) as well as in the content (units can be shown in the column header).
\end{enumerate}

\section{Formulas}

IJCAI's two-column format makes it difficult to typeset long formulas. A usual temptation is to reduce the size of the formula by using the {\tt small} or {\tt tiny} sizes. This doesn't work correctly with the current \LaTeX{} versions, breaking the line spacing of the preceding paragraphs and title, as well as the equation number sizes. The following equation demonstrates the effects (notice that this entire paragraph looks badly formatted, and the line numbers no longer match the text):
%
\begin{tiny}
    \begin{equation}
        x = \prod_{i=1}^n \sum_{j=1}^n j_i + \prod_{i=1}^n \sum_{j=1}^n i_j + \prod_{i=1}^n \sum_{j=1}^n j_i + \prod_{i=1}^n \sum_{j=1}^n i_j + \prod_{i=1}^n \sum_{j=1}^n j_i
    \end{equation}
\end{tiny}%

Reducing formula sizes this way is strictly forbidden. We {\bf strongly} recommend authors to split formulas in multiple lines when they don't fit in a single line. This is the easiest approach to typeset those formulas and provides the most readable output%
%
\begin{align}
    x = & \prod_{i=1}^n \sum_{j=1}^n j_i + \prod_{i=1}^n \sum_{j=1}^n i_j + \prod_{i=1}^n \sum_{j=1}^n j_i + \prod_{i=1}^n \sum_{j=1}^n i_j + \nonumber \\
    +   & \prod_{i=1}^n \sum_{j=1}^n j_i.
\end{align}%

If a line is just slightly longer than the column width, you may use the {\tt resizebox} environment on that equation. The result looks better and doesn't interfere with the paragraph's line spacing: %
\begin{equation}
    \resizebox{.91\linewidth}{!}{$
            \displaystyle
            x = \prod_{i=1}^n \sum_{j=1}^n j_i + \prod_{i=1}^n \sum_{j=1}^n i_j + \prod_{i=1}^n \sum_{j=1}^n j_i + \prod_{i=1}^n \sum_{j=1}^n i_j + \prod_{i=1}^n \sum_{j=1}^n j_i
        $}.
\end{equation}%

This last solution may have to be adapted if you use different equation environments, but it can generally be made to work. Please notice that in any case:

\begin{itemize}
    \item Equation numbers must be in the same font and size as the main text (10pt).
    \item Your formula's main symbols should not be smaller than {\small small} text (9pt).
\end{itemize}

For instance, the formula
%
\begin{equation}
    \resizebox{.91\linewidth}{!}{$
            \displaystyle
            x = \prod_{i=1}^n \sum_{j=1}^n j_i + \prod_{i=1}^n \sum_{j=1}^n i_j + \prod_{i=1}^n \sum_{j=1}^n j_i + \prod_{i=1}^n \sum_{j=1}^n i_j + \prod_{i=1}^n \sum_{j=1}^n j_i + \prod_{i=1}^n \sum_{j=1}^n i_j
        $}
\end{equation}
%
would not be acceptable because the text is too small.

\section{Examples, Definitions, Theorems and Similar}

Examples, definitions, theorems, corollaries and similar must be written in their own paragraph. The paragraph must be separated by at least 2pt and no more than 5pt from the preceding and succeeding paragraphs. They must begin with the kind of item written in 10pt bold font followed by their number (e.g.: {\bf Theorem 1}),
optionally followed by a title/summary between parentheses in non-bold font and ended with a period (in bold).
After that the main body of the item follows, written in 10 pt italics font (see below for examples).

In \LaTeX{} we strongly recommend that you define environments for your examples, definitions, propositions, lemmas, corollaries and similar. This can be done in your \LaTeX{} preamble using \texttt{\textbackslash{newtheorem}} -- see the source of this document for examples. Numbering for these items must be global, not per-section (e.g.: Theorem 1 instead of Theorem 6.1).

\begin{example}[How to write an example]
    Examples should be written using the example environment defined in this template.
\end{example}

\begin{theorem}
    This is an example of an untitled theorem.
\end{theorem}

You may also include a title or description using these environments as shown in the following theorem.

\begin{theorem}[A titled theorem]
    This is an example of a titled theorem.
\end{theorem}

\section{Proofs}

Proofs must be written in their own paragraph(s) separated by at least 2pt and no more than 5pt from the preceding and succeeding paragraphs. Proof paragraphs should start with the keyword ``Proof." in 10pt italics font. After that the proof follows in regular 10pt font. At the end of the proof, an unfilled square symbol (qed) marks the end of the proof.

In \LaTeX{} proofs should be typeset using the \texttt{\textbackslash{proof}} environment.

\begin{proof}
    This paragraph is an example of how a proof looks like using the \texttt{\textbackslash{proof}} environment.
\end{proof}


\section{Algorithms and Listings}

Algorithms and listings are a special kind of figures. Like all illustrations, they should appear floated to the top (preferably) or bottom of the page. However, their caption should appear in the header, left-justified and enclosed between horizontal lines, as shown in Algorithm~\ref{alg:algorithm}. The algorithm body should be terminated with another horizontal line. It is up to the authors to decide whether to show line numbers or not, how to format comments, etc.

In \LaTeX{} algorithms may be typeset using the {\tt algorithm} and {\tt algorithmic} packages, but you can also use one of the many other packages for the task.

\begin{algorithm}[tb]
    \caption{Example algorithm}
    \label{alg:algorithm}
    \textbf{Input}: Your algorithm's input\\
    \textbf{Parameter}: Optional list of parameters\\
    \textbf{Output}: Your algorithm's output
    \begin{algorithmic}[1] %[1] enables line numbers
        \STATE Let $t=0$.
        \WHILE{condition}
        \STATE Do some action.
        \IF {conditional}
        \STATE Perform task A.
        \ELSE
        \STATE Perform task B.
        \ENDIF
        \ENDWHILE
        \STATE \textbf{return} solution
    \end{algorithmic}
\end{algorithm}

\section{\LaTeX{} and Word Style Files}\label{stylefiles}

The \LaTeX{} and Word style files are available on the IJCAI--ECAI 26
website, \url{https://2026.ijcai.org/}.
These style files implement the formatting instructions in this
document.

The \LaTeX{} files are {\tt ijcai26.sty} and {\tt ijcai26.tex}, and
the Bib\TeX{} files are {\tt named.bst} and {\tt ijcai26.bib}. The
\LaTeX{} style file is for version 2e of \LaTeX{}, and the Bib\TeX{}
style file is for version 0.99c of Bib\TeX{} ({\em not} version
0.98i). 

The Microsoft Word style file consists of a single file, {\tt ijcai26.docx}. 
%This template differs from the one used for IJCAI--23.

These Microsoft Word and \LaTeX{} files contain the source of the
present document and may serve as a formatting sample.

Further information on using these styles for the preparation of
papers for IJCAI--ECAI 26 can be obtained by contacting {\tt
        proceedings@ijcai.org}.

\appendix

\section*{Ethical Statement}

There are no ethical issues.

\section*{Acknowledgments}

The preparation of these instructions and the \LaTeX{} and Bib\TeX{}
files that implement them was supported by Schlumberger Palo Alto
Research, AT\&T Bell Laboratories, and Morgan Kaufmann Publishers.
Preparation of the Microsoft Word file was supported by IJCAI.  An
early version of this document was created by Shirley Jowell and Peter
F. Patel-Schneider.  It was subsequently modified by Jennifer
Ballentine, Thomas Dean, Bernhard Nebel, Daniel Pagenstecher,
Kurt Steinkraus, Toby Walsh, Carles Sierra, Marc Pujol-Gonzalez,
Francisco Cruz-Mencia and Edith Elkind.


%% The file named.bst is a bibliography style file for BibTeX 0.99c
\bibliographystyle{named}
\bibliography{ijcai26}

\end{document}

